\documentclass[]{report}
\renewcommand\thesection{\arabic{section}}%for page numbering in arabics
\usepackage{graphicx,tabularx}%for figures and tables
\usepackage[utf8]{inputenc} %allows special characters such as ä, ö, ỳ
\usepackage[english]{babel}  %set the language to English
\usepackage[margin=1.5in]{geometry} %change page margins 
\usepackage{sectsty}%section headers
\allsectionsfont{\sffamily\large}
\subsectionfont{\sffamily\normalsize}
\linespread{1.2}% line distance
\usepackage{lipsum}% http://ctan.org/pkg/lipsum
\usepackage{caption}%use for captions on tables
%use this exact command. The style and bibliographystyle has to be authoryear (Havard). The sorting is nyt: name, year, title so that the bibliography is sorted alphabetically. firstinits=true shortens the names: Albert Einstein -> A. Einstein
\usepackage[backend=bibtex,style=authoryear,bibstyle=authoryear,sorting=nyt,firstinits=true]{biblatex}
\setlength\parindent{0pt}%include this so that your paragraphs don't indent automatically
\addbibresource{report.bib} %this attaches your bib-file, your bibliography (must be in the same folder)
\usepackage[compact]{titlesec}%include title formatting package

% Title Page
\title{Top Text}
\author{Tom Jacobs and James Boogaard}
\date{December 15th 2022 \\Module: SEAR \\Venlo, Limburg, Netherlands}


\begin{document}
	
	\maketitle
	
	\begin{abstract}
		This is the abstract.
		
		\pagenumbering{roman}
		
	\end{abstract}
	
	\tableofcontents
	\setcounter{page}{3}
	\listoffigures %UNCOMMENT IF YOU HAVE FIGURES
	%\listoftables %UNCOMMENT IF YOU HAVE TABLES
	\pagebreak
	
	\pagenumbering{arabic}	
	
	\section{Introduction}
	During the year of 2022 there has been a massive increase in the popularity of text-to-image generators, an artificial intelligence technique that translates human text into a computer-generated image. This has led to many different generators being made, which on the one hand has given artist, a person that uses a text-to-image generator to create an image with text, many more options but has also made the landscape difficult to navigate. These new options are great for the field as a whole, however it also means that artists who are not experienced may struggle to find the generator which suits their needs. This paper makes an effort to try and alleviate this problem by picking four of the most used text-to-image generators and comparing them based on their accuracy so we can help give some perspective on which text-to-image generator is best for them.
	
	A research paper was published on the nineteenth of February 2020 called: "A survey and taxonomy of adversarial neural networks for text-to-image synthesis" that makes use of surveys to compare models for text-to-image generations. This paper aimed to show which text-to-image technique yielded the most realistic results in a wide variety of categories. One issue with this paper was that is was written before the explosion of consumer text-to-image generators happened, which is why we will instead focus on modern consumer-grade generators which are available today.
	
	Our goal during the course of this reaserch paper is to find out what is the best text-to-image generator, out of the four we chose, for artists who are experienced as well as artists who are not experienced in artificial art generation.
	
	The best text-to-image generator out of the four we chose will be stable diffusion. This is due to the fact that stable diffusion offers its users more tools and techniques like inpainting (allows you to regenerate a specific area of an image without regenerating the whole image) which allows the user more freedom to fine-tume their result. Given enough time this will yield a more accurate result. 
	
	
	\section{Methods}
	\lipsum
	\section{Results}
	\newpage	
	\section{Discussion}
	
\end{document}


\printbibliography[title=References]

\end{document}          
       
